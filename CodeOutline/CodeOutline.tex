\documentclass[titlepage,12pt,letter]{article}
\usepackage{extsizes}			\usepackage{bm}
\usepackage[english]{babel}		\usepackage{amsmath}
\usepackage{amsfonts}			\usepackage{amssymb}
\usepackage{graphicx}			\usepackage{epsfig}
\usepackage{sectsty}			\usepackage{booktabs}
\usepackage{floatrow}			\floatsetup[table]{capposition=top}	
\usepackage{siunitx}			\usepackage{caption}
\usepackage{subcaption}			\usepackage{listings}
\usepackage{authblk}			\usepackage{rotating}
\usepackage{multicol}			\usepackage{multirow}
\usepackage{mathrsfs}			\usepackage[useregional]{datetime2}
\PassOptionsToPackage{hyphens}{url}
\usepackage{hyperref}			\hypersetup{colorlinks = true, citecolor = OliveGreen}
\usepackage[dvipsnames]{xcolor}	\usepackage{pdfpages}
\usepackage{soul}				\usepackage{enumitem}	
\bibliographystyle{apalike}
\usepackage{indentfirst}
\usepackage{pdflscape}
\usepackage{arydshln}
%\usepackage[numbered,framed]{mcode}

\def\thesection{\arabic{section}}	\def\theequation{\arabic{equation}}

\setlength{\textwidth}{6.5in}		\setlength{\textheight}{9.0in}
\setlength{\oddsidemargin}{0in}		\setlength{\topmargin}{-0.5in}

%% New Commands
\newcommand{\p}{\partial}
\newcommand{\e}[1]{\ensuremath{\times 10^{#1}}}
\newcommand{\abs}[1]{\left| #1 \right|}
\newcommand{\paren}[1]{\left( #1 \right)}
\newcommand{\bracket}[1]{\left[ #1 \right]}
\newcommand{\cbracket}[1]{\left\{ #1 \right\}}
\newcommand{\at}[2]{\left. #1 \right|_{#2}}
\newcommand{\highlightbox}[2]{\colorbox{#1}{$\displaystyle #2$}}
\newcommand{\hlcolor}[2]{\sethlcolor{#1}\hl{#2}}
\newcommand{\F}{^\circ F}
\newcommand{\C}{^\circ C}
\newcommand{\inch}{^{\prime \prime}}
\newcommand{\pt}[1][]{\frac{\p #1}{\p t}} % partial time derivative
\newcommand{\Dt}[1][]{\frac{{\rm D} #1}{{\rm D} t}} % material derivative
\newcommand{\vint}[1]{\int_{V} #1 {\rm \ d}V} % Volume integral
\newcommand{\aint}[1]{\int_{A} #1 {\rm \ d}A} % Area integral
\newcommand{\bbar}[1]{\overline{\overline{#1}}} % double over bar for tensor
\newcommand{\dd}{\,{\rm d}}
\newcommand{\mathbold}[1]{ {\bm #1} }
\newcommand{\re}{Re}
\newcommand{\retau}{ {Re}_{\tau} }
\newcommand{\norm}[1]{\left\lVert#1\right\rVert}

\newcommand{\red}[1]{{{\color{red}\textbf{#1}}}}
\newcommand{\blue}[1]{{{\color{blue}\textbf{#1}}}}
\newcommand{\green}[1]{{{\color{green}\textbf{#1}}}}

\newcommand{\question}{{{\color{red}\textbf{?}}}}

%% Table
\usepackage{array}
\newcolumntype{L}[1]{>{\raggedright\let\newline\\\arraybackslash\hspace{0pt}}m{#1}}
\newcolumntype{C}[1]{>{\centering\let\newline\\\arraybackslash\hspace{0pt}}m{#1}}
\newcolumntype{R}[1]{>{\raggedleft\let\newline\\\arraybackslash\hspace{0pt}}m{#1}}
%% equation, figure and tabel number with section number
\usepackage{chngcntr}
\counterwithin{figure}{section}
\counterwithin{table}{section}
\numberwithin{equation}{section}
%% These settings can be changed
\setlength{\parindent}{0in}
\setlength{\parskip}{0.5em}
\setcounter{section}{0}


\graphicspath{{D:/Academic/Caltech/Research/}}
%\epstopdfsetup{verbose=false}

\begin{document}
{\centering
{\huge\textbf{DNS Code Outline} \par}

\vspace{0.2in}
Yuting Huang\par
\today\par}

\section{DNS Parameters and Checklist}
\subsection{DNS Parameters}


\subsection{DNS Checklist}
\subsubsection{Important Paramters}
\begin{table}[H]
	\centering 
	\renewcommand{\arraystretch}{1.4} 
	\begin{tabular}{l|l|l}
		File&Line& \\ \hline
		\multirow{2}{*}{save\_flowfield.f90}&32&\textbf{SamplingFrequencyInTimesteps} (fix step sampling)\\
		&33-35&\textbf{samplingPeriod} (fix time sampling)\\ \hline
		
		\multirow{2}{*}{cross.f}&235&fix time sampling\\
		&236&fix step sampling\\ \hline
		
		\multirow{2}{*}{wall\_roughness.f90}&7-8&\textbf{disturbanceU}, \textbf{disturbanceV}\\
		&11-12&\textbf{targetWavenumberX}, \textbf{absoluteValueTargetWavenumberZ} \\ \hline
		
		\multirow{6}{*}{launch.sh}&9&\textbf{runName}\\
		&10-11&\textbf{runTime}, \textbf{jobDependency}\\
		&15&\textbf{inputFile}\\
		&19&\textbf{Re}\\
		&21,23,26&\textbf{nimag}, \textbf{nstep}, \textbf{nhist}\\
		&28&\textbf{CFL}\\ \hline		
	\end{tabular} 
\end{table} 

\subsubsection{Directory Settings}
\begin{table}[H]
	\centering 
	\renewcommand{\arraystretch}{1.4} 
	\begin{tabular}{l|l|l}
		File&Line& \\ \hline
		\multirow{2}{*}{launch.sh}&28&\textbf{runFolder}\\
		&39&\textbf{scratchFolder}\\ \hline		
	\end{tabular} 
\end{table} 

\subsubsection{Processor Number}
\begin{table}[H]
	\centering 
	\renewcommand{\arraystretch}{1.4} 
	\begin{tabular}{l|l|l}
		File&Line& \\ \hline
		ctes3D&15&\textbf{numerop}\\
		launch.sh&13&\textbf{mpiProcessors}\\ \hline		
	\end{tabular} 
\end{table} 


\section{Governing Equations}
Rewrite the momentum equations in velocity-vorticity form
\begin{equation}
	\frac{\partial \bm{u}}{\partial t}+\frac{1}{2} \nabla(\bm{u} \cdot \bm{u})-\bm{H}=-\nabla p+\frac{1}{\re} \nabla^{2} \bm{u} \label{eq:nse}
\end{equation}

where $\bm{H}$ contains the rotational part of the convection term
\begin{equation*}
	\bm{H}=\bm{u} \times \bm{\omega}
\end{equation*}

An evolution equation for the vorticity and velocity vectors can be obtained
\begin{align*}
	\frac{\partial \bm{\omega}}{\partial t}-\nabla \times \bm{H}&=\frac{1}{\re} \nabla^{2} \bm{\omega}\\
	\frac{\partial}{\partial t} \nabla^{2} \bm{u}+\nabla \times(\nabla \times \bm{H})&=\frac{1}{\re} \nabla^{4} \bm{u}
\end{align*}

For $g = \omega_2$ and $\phi = \nabla^2 v$, the evolution equations are
\begin{align}
	\frac{\partial}{\partial t} g &=h_{g}+\frac{1}{\re} \nabla^{2} g \label{eq:vor}\\
	\frac{\partial}{\partial t} \phi &=h_{v}+\frac{1}{\re} \nabla^{2} \phi \label{eq:phi}
\end{align}
and the nonlinear terms are collected in
\begin{align}
	h_{g} &=\frac{\partial H_{1}}{\partial z}-\frac{\partial H_{3}}{\partial x} \\
	h_{v} &=-\frac{\partial}{\partial y}\left(\frac{\partial H_{1}}{\partial x}+\frac{\partial H_{3}}{\partial z}\right)+\left(\frac{\partial^{2}}{\partial x^{2}}+\frac{\partial^{2}}{\partial z^{2}}\right) H_{2}
\end{align}
Importantly note the mean pressure gradient contained in $\bm{H}$ vanishes due to the spatial derivatives.

\subsection{Discretization}
The first and second derivative in $y$ can be summarized as:
\begin{align}
	\bm{D}_{11} \cdot u' &= \bm{D}_{12} \cdot u \label{eq:fd1}\\
	\bm{D}_{21} \cdot u'' &= \bm{D}_{22} \cdot u \label{eq:fd2}
\end{align}
where $\bm{D}_{11}$, $\bm{D}_{12}$, $\bm{D}_{21}$, $\bm{D}_{22}$ are all penta-diagonal matrices.

We will use the additive semi-implicit RK schemes and treat the stiff linear operator $L$ implicitly and the non-stiff nonlinear operator $N$ explicitly. The RK scheme advances in three sub-steps from $t^{j}$ to $t^{j+1}$ and is defined as
\begin{align*}
	&\bm{q}^{s=2}=\bm{q}^{j}+\Delta t^{j+1}\left[\alpha_{1} L \bm{q}^{s=2}+\gamma_{1} N\left(\bm{q}^{j}\right)\right] \\
	&\bm{q}^{s=3}=\bm{q}^{j}+\Delta t^{j+1}\left[\alpha_{2} L \bm{q}^{s=3}+\gamma_{2} N\left(\bm{q}^{s=2}\right)\right] \\
	&\bm{q}^{j+1}=\bm{q}^{j}+\Delta t^{j+1}\left[\alpha_{3} L \bm{q}^{j+1}+\gamma_{3} N\left(\bm{q}^{s=3}\right)\right]
\end{align*}

where the superscript $s=2$ and $s=3$ label the intermediate solution at the second and third RK stage, respectively, and the $\alpha_{i}$ and $\gamma_{i}$ are the scheme parameters 
\begin{equation}
	\alpha = \gamma=\bracket{\frac{1}{3}, \ \frac{1}{2}, \ 1}
\end{equation}

\subsection{Wall-Normal Vorticity Equation}
The semi-implicit RK scheme applied to the ODE system for the Fourier transformed wall-normal vorticity~\eqref{eq:vor} multiplied by $\bm{D}_{21}$ and using~\eqref{eq:fd2} gives
\begin{equation*}
	\bm{D}_{21} \hat{\bm{g}}^{s+1}=\bm{D}_{21} \hat{\bm{g}}^{j}+\gamma_{s} \Delta t \bracket{ -\frac{1}{\re}\left(k_x^{2}+k_z^{2}\right) \bm{D}_{21} \hat{\bm{g}}^{s+1}+\frac{1}{\re} \bm{D}_{22} \hat{\bm{g}}^{s+1}+\bm{D}_{21} \hat{\bm{h}}_{g}^{s} }
\end{equation*}
Note that at the first RK stage $s=j$ and at the third stage $s+1=j+1$. Collecting all terms evaluated at the new time stage on the left-hand side gives
\begin{equation}
	\bracket{\bm{D}_{22}-\left(k_x^{2}+k_z^{2}+\frac{\re}{\gamma_{s} \Delta t}\right) \bm{D}_{21}} \hat{\bm{g}}^{s+1}=-\frac{\re}{\gamma_{s} \Delta t} \bm{D}_{21} \bracket{ \hat{\bm{g}}^{j}+\gamma_{s} \Delta t \hat{\bm{h}}_{g}^{s} } \label{eq:ome}
\end{equation}

The boundary conditions on $\hat{\bm{g}}$ are directly imposed by replacing the first and last line of the equation.

\subsection{Wall-Normal Velocity Equation}
Similarly, for $\phi$ we have
\begin{equation}
	\bracket{\bm{D}_{22}-\left(k_x^{2}+k_z^{2}+\frac{\re}{\gamma_{s} \Delta t}\right) \bm{D}_{21}} \hat{\bm{\phi}}_p^{s+1}=-\frac{\re}{\gamma_{s} \Delta t} \bm{D}_{21} \bracket{ \hat{\bm{\phi}}_p^{j}+\gamma_{s} \Delta t \hat{\bm{h}}_{v}^{s} } \label{eq:phip}
\end{equation}

To enforce the boundary conditions for $v$ and $\partial v / \partial y$ we decompose the wall-normal velocity into a particular solution $\hat{\bm{v}}_{p}^{s+1}$ and two homogeneous solutions $\hat{\bm{v}}_{1}^{s+1}$ and $\hat{\bm{v}}_{2}^{s+1}$
\begin{equation*}
\hat{\bm{v}}^{s+1}=\hat{\bm{v}}_{p}^{s+1}+A \hat{\bm{v}}_{1}^{s+1}+B \hat{\bm{v}}_{2}^{s+1}
\end{equation*}
with complex coefficients $A$ and $B$. This decomposition allows to enforce the boundary conditions on $v$ by means of the particular solution $\hat{\bm{v}}_{p}$ and the ones on $\partial v / \partial y$ by properly choosing the coefficients $A$ and $B$. 

Particular solutions of $\phi$ satisfy:
\begin{equation}
	\bracket{\bm{D}_{22}-\left(k_x^{2}+k_z^{2}+\frac{\re}{\gamma_{s} \Delta t}\right) \bm{D}_{21}} \hat{\bm{\phi}}_{1,2}^{s+1}=0 \label{eq:phi1}
\end{equation}


The equation for  $\hat{\bm{v}}_{p}^{s+1}$ and two homogeneous solutions $\hat{\bm{v}}_{1}^{s+1}$ and $\hat{\bm{v}}_{2}^{s+1}$ is
\begin{equation}
		\bracket{\bm{D}_{22}-\left(k_x^{2}+k_z^{2}\right) \bm{D}_{21}} \hat{\bm{v}}_{p,1,2}^{s+1}=\bm{D}_{21} \hat{\bm{\phi}}_{p,1,2}^{s+1} \label{eq:v}
\end{equation}

The boundary conditions for the particular solution are
\begin{equation}
	\left\{
\begin{aligned}
	&\hat{\bm{\phi}}_{p}^{s+1}\left(y_{1}\right)&&=0, &&\quad &&\hat{\bm{\phi}}_{p}^{s+1}\left(y_{N}\right)&&=0 \\
	&\hat{\bm{v}}_{p}^{s+1}\left(y_{1}\right)&&=\hat{{v}}_{b}^{s+1}, &&\quad &&\hat{\bm{v}}_{p}^{s+1}\left(y_{N}\right)&&=\hat{{v}}_{t}^{s+1}
\end{aligned}\right.\label{eq:phip_bc}
\end{equation}
and the homogeneous solutions satisfy
\begin{equation}
	\left\{\begin{aligned}
	&\hat{\bm{\phi}}_{1}^{s+1}\left(y_{1}\right)&&=1, &&\quad &&\hat{\bm{\phi}}_{1}^{s+1}\left(y_{N}\right)&&=0 \\
	&\hat{\bm{v}}_{1}^{s+1}\left(y_{1}\right)&&=0, &&\quad &&\hat{\bm{v}}_{1}^{s+1}\left(y_{N}\right)&&=0
\end{aligned}\right. \label{eq:phi1_bc}
\end{equation}
\begin{equation}
 \left\{	\begin{aligned}
	&\hat{\bm{\phi}}_{2}^{s+1}\left(y_{1}\right)&&=0, &&\quad &&\hat{\bm{\phi}}_{2}^{s+1}\left(y_{N}\right)&&=1 \\
	&\hat{\bm{v}}_{2}^{s+1}\left(y_{1}\right)&&=0, &&\quad &&\hat{\bm{v}}_{2}^{s+1}\left(y_{N}\right)&&=0
\end{aligned}\right. \label{eq:phi2_bc}
\end{equation}

Due to the symmetry of the channel, we have:
\begin{equation*}
	\hat{\bm{\phi}}_1(y) = \hat{\bm{\phi}}_2(-y)
\end{equation*}
and therefore:
\begin{equation*}
	\hat{\bm{v}}_1(y) = \hat{\bm{v}}_2(-y)
\end{equation*}
and
\begin{equation*}
	\hat{\bm{v}}'_1(y) = -\hat{\bm{v}}'_2(-y)
\end{equation*}

The constants $A$ and $B$ are then chosen such that the boundary conditions on $\partial v / \partial y$ are satisfied

\begin{equation*}
	\renewcommand{\arraystretch}{2}
\begin{bmatrix}
	\hat{v}'_1\left(y_{1}\right) & \hat{v}'_2\left(y_{1}\right) \\
	\hat{v}'_1\left(y_{N}\right) & \hat{v}'_2\left(y_{N}\right)
\end{bmatrix}
\begin{bmatrix}
	A\\ B
\end{bmatrix}
=
\begin{bmatrix}
	\hat{v}'_b-\hat{v}'_p\left(y_{1}\right) \\
	\hat{v}'_t-\hat{v}'_p\left(y_{N}\right) \\
\end{bmatrix}
\end{equation*}

Using the symmetry relationships
\begin{equation*}
	\renewcommand{\arraystretch}{2}
	\begin{bmatrix}
		\hat{v}'_1\left(y_{1}\right) & -\hat{v}'_1\left(y_{N}\right) \\
		\hat{v}'_1\left(y_{N}\right) & -\hat{v}'_1\left(y_{1}\right)
	\end{bmatrix}
	\begin{bmatrix}
		A\\ B
	\end{bmatrix}
	=
	\begin{bmatrix}
		\hat{v}'_b-\hat{v}'_p\left(y_{1}\right) \\
		\hat{v}'_t-\hat{v}'_p\left(y_{N}\right) \\
	\end{bmatrix}
\end{equation*}

and invert
\begin{equation}
	\renewcommand{\arraystretch}{2}
	\begin{bmatrix}
		A\\ B
	\end{bmatrix}
	=\frac{1}{-\hat{v}'_1(y_1)\cdot \hat{v}'_1(y_1)+ \hat{v}'_1(y_N)\cdot \hat{v}'_1(y_N)}
	\begin{bmatrix}
		-\hat{v}'_1\left(y_{1}\right)& \hat{v}'_1\left(y_{N}\right) \\
		-\hat{v}'_1\left(y_{N}\right) &	\hat{v}'_1\left(y_{1}\right) 
	\end{bmatrix}
	\begin{bmatrix}
	\hat{v}'_b-\hat{v}'_p\left(y_{1}\right) \\
	\hat{v}'_t-\hat{v}'_p\left(y_{N}\right) \\
\end{bmatrix} \label{eq:AB}
\end{equation}

\subsection{00 Modes}
The 00 modes of $u$ and $w$ need to be computed separately, as they contribute to the mass flux, which is held constant. 

In addition, the equations used for computing $u$, $w$, $\omega_1$ and $\omega_3$ from $\phi$ and $\omega_2$ are singular when $k_x = k_z = 0$, therefore requiring special treatment. We also have:
\begin{align*}
	\omega_{1,00} &= +\frac{\p}{\p y} w_{00}\\
	\omega_{3,00} &= -\frac{\p}{\p y} u_{00}
\end{align*}
and also:
\begin{equation}
	v_{00} = \phi_{00} = \omega_{2,00} = 0 \label{eq:00}
\end{equation}

Equation~\eqref{eq:nse} can be rewritten for $u_{00}$ and $w_{00}$, the $k_x = k_z = 0$ modes:
\begin{align*}
	\frac{\p}{\p t} u_{00}  &= H_{1,00} + \frac{1}{\re} \frac{\p^2}{\p y^2} u_{00} + C_1\\
	\frac{\p}{\p t} w_{00}  &= H_{3,00} + \frac{1}{\re} \frac{\p^2}{\p y^2} w_{00} + C_2
\end{align*} 
where the constant terms come from the mean pressure gradient. Apply the semi-implicit RK scheme multiplied by $\bm{D}_{21}$ without the constants and using~\eqref{eq:fd2} gives
\begin{align*}
	\bm{D}_{21} u_{00}^{s+1} &= \bm{D}_{21} u_{00}^j + \gamma_s \Delta t \bracket{\frac{1}{\re} \bm{D}_{22} u_{00}^{s+1} +  \bm{D}_{21} H_{1,00}^s}\\
	\bm{D}_{21} w_{00}^{s+1} &= \bm{D}_{21} w_{00}^j + \gamma_s \Delta t \bracket{\frac{1}{\re} \bm{D}_{22} w_{00}^{s+1} +  \bm{D}_{21} H_{3,00}^s}
\end{align*}
rearrange:
\begin{align}
	\bracket{ \bm{D}_{22} - \frac{\re}{\gamma_s \Delta t}\bm{D}_{21} } u_{00,p}^{s+1}&= - \frac{\re}{\gamma_s \Delta t}\bm{D}_{21} \bracket{u_{00}^j +   \gamma_s \Delta t H_{1,00}^s} \label{eq:u00p}\\
	\bracket{ \bm{D}_{22} - \frac{\re}{\gamma_s \Delta t}\bm{D}_{21} } w_{00,p}^{s+1}&= - \frac{\re}{\gamma_s \Delta t}\bm{D}_{21} \bracket{w_{00}^j +    \gamma_s \Delta t H_{3,00}^s} \label{eq:w00p}
\end{align}
To enforce constant mass flux, we will find the solution to a constant forcing:
\begin{align}
	\bracket{ \bm{D}_{22} - \frac{\re}{\gamma_s \Delta t}\bm{D}_{21} } u_{00,h}^{s+1}&= - \frac{\re}{\gamma_s \Delta t}\bm{D}_{21} \label{eq:u00h}\\
	\bracket{ \bm{D}_{22} - \frac{\re}{\gamma_s \Delta t}\bm{D}_{21} } w_{00,h}^{s+1}&= - \frac{\re}{\gamma_s \Delta t}\bm{D}_{21} \label{eq:w00h}
\end{align}
Then compute the appropriate constants
\begin{align}
	 u_{00}^{s+1}&= u_{00,p}^{s+1} + C_1 u_{00,h}^{s+1}\\
	 w_{00}^{s+1}&= w_{00,p}^{s+1} + C_2 w_{00,h}^{s+1}
\end{align}
so that the desired constant mass flux is satisfied.

\begin{figure}[H]
	\centering
	\includegraphics[width=0.75\textwidth]{DNS3.png}
\end{figure}

\begin{figure}[H]
	\centering
	\includegraphics[width=0.75\textwidth]{DNS4.png}
\end{figure}


%\section{Initialization}
%
%The simulation starts by reading the initial data $\hat{g}_{k}\left(t^{0}\right)$ and $\hat{\phi}_{k}\left(t^{0}\right)$ from the restart file of a previous simulation run. The initial wall-normal velocity $\hat{\bm{v}}_{k}\left(t^{0}\right)$ is not read, but instead calculated from $\hat{\phi}_{k}\left(t^{0}\right)$ using (2.77). Note that in contrast to the regular timesteps, $\hat{\phi}_{k}\left(t^{0}\right)$ is given by the restart file and cannot be decomposed according to (2.105). This makes a decomposition of $v$ in the spirit of $(2.99)$ impossible at the first timestep and therefore $\hat{\bm{v}}_{k}^{j}=\hat{\bm{v}}_{k, p}^{j}$, i.e. only a particular solution $\hat{\bm{v}}_{k, p}^{j}$ satisfying $\hat{\bm{v}}_{k, b}^{j}$ and $\hat{v}_{k, t}^{j}$ is calculated from $2.100$ at simulation startup.






\newpage


\section{Global Variables}
\begin{table}[H]
	\centering 
	\renewcommand{\arraystretch}{1.5} 
	\begin{tabular}{c|c|c|m{7cm}}
		Variable&Value&Size&Description \\ \hline
		alp, bet& 0.5, 1.0&& Lowest $k_x$, $k_z$ wavenumbers\\ \hline
		mgalx, mgalz, my& 768, 768, 232&& Number of physical space coordinates \\
		mx, mz& 512, 511&& Number of wavenumbers\\ \hdashline
		mgalx1, mgalz1& mgalx-1, mgalz-1&&\\
		mx1, my1, mz1&mx/2-1, my-1, mz-1&&\\ \hdashline
		jb, je& & &$y$ start and end index for this processor \\
		kb, ke& & &$k_z$ start and end index for this processor \\
		mmy, mmz&je-jb+1, ke-kb+1& &number of $y$ and $k_z$ grid points for this processor\\
		nbuffsize&mx*max(mmy*mz,mmz*my)&&buffer size for \textbf{phi}, \textbf{vor}, \textbf{phiwk}, \textbf{vorwk}, \textbf{dvordy}, \textbf{hv}, \textbf{hg}, which are matrices that involves a change between $k_x-k_z$ plane and $k_x-y$ planes\\ \hline		
		xalp&	&0:mx1	&Complex streamwize wavenumber $ik_x$\\			
		xbet&	&0:mz1	&Complex spanwize wavenumber $ik_z$\\		\hline
		alp2& - xalp$^2$ &0:mx1& $k_x^2$ (Note that xalp contains $i$)\\
		bet2& - xbet$^2$ &0:mx1& $k_z^2$ (Note that xbet contains $i$)\\ \hline
		trp, trp2&	&1:my&Trapezoid integration in y, divided by channel height 2 (the two are different precision?)\\ \hline
		c & [1/3, 1/2, 1]&1:3& RK constants \\
		r1& c(rkstep)*Deltat& &\\
		dtr& Re/Deltat& & \\
		dtr1& dtr/c(rkstep) = Re/r1 & & \\ \hline		
		dt11, dt12&&(5, my)&coefficients for 5 diag matrix $DT11$, $DT12$:	$DT11\cdot u' = DT12\cdot u$ \\
		dt21, dt22&&(5, my)&coefficients for 5 diag matrix $DT21$, $DT22$: $DT21\cdot u'' = DT22\cdot u$ \\ 
		prem1& dt11 through bandec5&(5, my) &\\\hline
		
	\end{tabular} 
\end{table} 

\section{Low Level Functions}
\begin{table}[H]
	\centering 
	\renewcommand{\arraystretch}{1.4} 
	\begin{tabular}{c|c|c|m{6.7cm}}
		Function&Inputs&Outputs&Description \\ \hline
		deryr(uwk,fwk,n)& \textbf{uwk} & \textbf{fwk} & First derivative in y. uwk and fwk are size my. n is not used.\\
		deryr2(u,du,m,n,wk4)& \textbf{u} & \textbf{du} & First derivative in y. u and du are size 2,my,m, where first index describe real/complex. Input n should be my.\\ \hline
		fourxz(fou,phys,-1,mmy)&\textbf{phys}& \textbf{fou} & FFT\\
		fourxz(fou,phys,+1,mmy)&\textbf{fou}& \textbf{phys} & IFFT\\ \hline		
		bandec5(a,n); banbks5(a,n,b)&\textbf{a},\textbf{n},\textbf{b}&\textbf{b}& compute $b = A\backslash b$, where $A$ is 5-diagonal $n\times n$ matrix with diagonals given by the rows of a, with the first row as the lowest diagonal. 
		
		bandec5 updates \textbf{a}, then banbks5 updates \textbf{b} to the final answer\\ \hline
		\multirow{2}{*}{Lapv1(f2,fwk,rK,bcb,bct)}&\textbf{f2} ($f$), \textbf{rK} ($a$)& \multirow{2}{*}{\textbf{fwk} ($u$)}  &solve: $u'' - a\cdot u = f$, real part only	\\
		&\textbf{bcb}, \textbf{bct}& & with $u(-1)= bcb$, $u(+1)= bct$\\ \hdashline
		
		\multirow{2}{*}{Lapvdv(phi,v,dvdy,rK,bcb,bct)}&\textbf{phi} ($\phi$), \textbf{rK} ($a$)& {\textbf{v}}  &solve: $v'' - a\cdot v = \phi$, real and imaginary, then compute the first derivative	\\
		&\textbf{bcb}, \textbf{bct}&\textbf{dvdy} & with $v(-1)= bcb$, $v(+1)= bct$\\ \hdashline
		
		\multirow{2}{*}{Lapv(phi,v,rK,bcb,bct)}&\textbf{phi} ($\phi$), \textbf{rK} ($a$)& {\textbf{v}}  &same as Lapvdv, without computing\\
		&\textbf{bcb}, \textbf{bct}& & first derivative\\ \hdashline
		
	
		\multicolumn{3}{c|}{Bilap(phi,v,dvdy,f,rk1,rk2,bcbv,bctv,bcbdv,bctdv)} &same as lapsov, without solving for omega, see next section for lapsov\\ \hline

	\end{tabular} 
\end{table} 

\newpage
\section{Main Functions}
\subsection{lapsov}
Defined on line 641 - 1008 in laps.v7.f

Used on line 758 - 760 in cross.f

Used for computing solution of $\phi, v, \omega_2$ at the next RK stage.

\paragraph{Input/Outputs:}\phantom{a}\\
lapsov(phi,v,dvdy,f,ome,g, rk1,rk2,bcbv,bctv,bcbdv,bctdv,bcbo,bcto)

\begin{table}[H]
	\centering 
	\renewcommand{\arraystretch}{1.4} 
	\begin{tabular}{c|c|c|c}
		&Variable name &           &Variable name        \\
		&in lapsov       &Description&when called in cross \\ \hline
		\multirow{6}{*}{Input}&\textbf{f, g}&forcing for $\phi$ and $\omega$: $h_v$ and $h_g$&\textbf{hv(0,i,k)},\textbf{hg(0,i,k)}\\
		&\textbf{rk1}&constant, should be $k_x^2+k_z^2+\re/(\gamma_s \Delta t)$ &\textbf{rk2}\\
		&\textbf{rk2}&constant, should be $k_x^2+k_z^2$ &\textbf{rk}\\ 
		&\textbf{bcbv, bctv}&boundary conditions for $v$&\textbf{bcb,bct}\\
		&\textbf{bcbdv, bctdv}&boundary conditions for $dv/dy$&\textbf{bcbdv,bctdv}\\
		&\textbf{bcbo, bcto}&boundary conditions for $\omega$&\textbf{bcbo,bcto}\\
		
		\hline
		
		\multirow{4}{*}{Output}&\textbf{phi}& $\phi = \nabla^2 v$&\textbf{phiwk(0,i,k)}\\
		&\textbf{v}&$v$&\textbf{hg(0,i,k)}\\
		&\textbf{dvdy}&$dv/dy$&\textbf{hv(0,i,k)}\\
		&\textbf{ome}&$\omega_2$&\textbf{vorwk(0,i,k)}\\\hline
	\end{tabular} 
\end{table} 
All output variables have size (1:2, 1:my), where 1 in the first index indicates real part and 2 in the first index indicate imaginary part.

\begin{table}[H]
	\centering 
	\renewcommand{\arraystretch}{1.5} 
	\begin{tabular}{l|l}
		Description& Line \\ \hline
		Prepare \textbf{wk1} = $\bracket{DT22 - rk1 \cdot DT21}^{-1}$ &719-738\\
		Solve for \textbf{phip} ($\phi_p$), real and imag, \eqref{eq:phip}, \eqref{eq:phip_bc}&741-778\\
		Solve for \textbf{phi1} ($\phi_1$), only real, \eqref{eq:phi1}, \eqref{eq:phi1_bc}&781-788\\
		Solve for \textbf{ome}  ($\omega$), real and imag, \eqref{eq:ome}, with BC given by \textbf{bcbo, bcto}&791-828\\
		\hdashline
		Prepare \textbf{wk1} = $\bracket{DT22 - rk2 \cdot DT21}^{-1}$ &834-853\\
		Solve for \textbf{vp} ($v_p$), real and imag, \eqref{eq:v}, \eqref{eq:phip_bc}, with BC given by \textbf{bcbv, bctv}&856-887\\
		Solve for \textbf{v1} ($v_1$), only real, \eqref{eq:v}, \eqref{eq:phi1_bc}&890-905\\
		\hdashline
		Compute \textbf{dvp}, the derivative of \textbf{vp}&908-946\\
		Compute \textbf{dv1}, the derivative of \textbf{v1}&949-968\\
		\hdashline
		Compute \textbf{A, B}, \eqref{eq:AB} so that the BC for $dv/dy$ given by \textbf{bcbdv, bctdv} are satisfied&971-987\\
		\hdashline
		Compute outputs \textbf{ome, phi, v, dvdy} using \textbf{A, B}&990-1003\\ \hline
	\end{tabular} 
\end{table}

Note: if \textbf{rk2} = 0 ($k_x = k_z = 0$), then \textbf{phi}, \textbf{ome}, \textbf{v} and \textbf{dvdy} are all set to 0, consistent with equation~\eqref{eq:00}.


\subsection{hvhg}
Defined on line 935 - 1693 in cross.f

Used on line 595 - 598 in cross.f

Used for computing the non-linear forcing 

\paragraph{Input/Outputs:}\phantom{a}\\
hvhg(phic,ome2c,rhvc,rhgc, rf0u,rf0w,ome1c, work2,sp,myid,rkstep, \\\phantom{hvhg} u1r,u2r,u3r,o1r,o2r,o3r, u1c,u2c,u3c,o1c,o2c,o3c )


\begin{table}[H]
	\centering 
	\renewcommand{\arraystretch}{1.4} 
	\begin{tabular}{c|c|c|c}
		&Variable name &           &Variable name        \\
		&in hvhg       &Description&when called in cross \\ \hline
		\multirow{6}{*}{Input}&\textbf{phic}&$\nabla^2v$&\textbf{phiwk}\\
		&\textbf{ome2c}&$\omega_2$&\textbf{vorwk}\\
		&\textbf{rhvc}&$dv/dy$&\textbf{hv}\\
		&\textbf{rhgc}&$v$&\textbf{hg}\\
		&\textbf{ome1c}&$d \omega_2/dy$ &\textbf{dvordy}\\ 
		&\textbf{work2}& 00 modes of $-\omega_3$, $\omega_1$, $u$, $w$ stacked in a column&\textbf{work}\\
		
		\hline
		
		\multirow{5}{*}{Output}&\textbf{rhvc}& $\paren{\frac{\p^2}{\p x^2}+\frac{\p^2}{\p z^2}} H_2$&\textbf{hv}\\
		&\textbf{rhgc}&$-\frac{\p H_3}{\p x} + \frac{\p H_1}{\p z}$&\textbf{hg}\\
		&\textbf{rf0u}&$H_1(0,0)$&\textbf{rf0u}\\
		&\textbf{rf0w}&$H_3(0,0)$&\textbf{rf0w}\\
		&\textbf{ome1c}&$\frac{\p H_1}{\p x} + \frac{\p H_3}{\p z}$&\textbf{dvordy}\\
		
		\hline
		
		\multirow{3}{*}{Other}&\textbf{sp}&Energy spectra, updated&\\
		&\textbf{myid}&Processor ID&\\
		&\textbf{rkstep}&RK step number&\\
		
		\hline
		
		\multirow{4}{*}{Temp}&\textbf{u1r,u2r,u3r}&\multirow{4}{6cm}{These variables are not accessed outside function hvhg}&\textbf{u1r,u2r,u3r}\\
		&\textbf{o1r,o2r,o3r}&&\textbf{o1r,o2r,o3r}\\
		&\textbf{u1c,u2c,u3c}&&\textbf{u1r,u2r,u3r}\\
		&\textbf{o1c,o2c,o3c}&&\textbf{o1r,o2r,o3r}\\
		
		\hline
		
	\end{tabular} 
\end{table} 

\begin{table}[H]
	\centering 
	\renewcommand{\arraystretch}{1.4} 
	\begin{tabular}{c|c|c|c|c}
		&Variable name &           &Variable name        &\\
		&in hvhg       &Size in hvhg&in cross& Size in cross\\ \hline
		\multirow{6}{*}{Input}&\textbf{phic} &complex(0:mx1,0:mz1,jb:je)&\textbf{phiwk}&real(0:2*my-1,0:mx1,kb:ke)\\
		&\textbf{ome2c}&complex(0:mx1,0:mz1,jb:je)&\textbf{vorwk}&\\
		&\textbf{rhvc}&complex(0:mx1,0:mz1,jb:je)&\textbf{hv}&\\
		&\textbf{rhgc}&complex(0:mx1,0:mz1,jb:je)&\textbf{hg}&\\
		&\textbf{ome1c}&complex(0:mx1,0:mz1,jb:je)&\textbf{dvordy}&\\ 
		&\textbf{work2}&real(1:4*my)&\textbf{work}&\\
		
		\hline
		
		\multirow{5}{*}{Output}&\textbf{rhvc}&complex(0:mx1,0:mz1,jb:je)&\textbf{hv}&\\
		&\textbf{rhgc}&complex(0:mx1,0:mz1,jb:je)&\textbf{hg}&\\
		&\textbf{rf0u}&real(1:my)&\textbf{rf0u}&\\
		&\textbf{rf0w}&real(1:my)&\textbf{rf0w}&\\
		&\textbf{ome1c}&complex(0:mx1,0:mz1,jb:je)&\textbf{dvordy}&\\ \hline
		\end{tabular} 
\end{table} 

All matrix input \textbf{phic, ome2c, rhvc, rhgc, ome1c} and matrix output \textbf{rhvc, rhgc, ome1c} are Fourier x - Fourier z - Physical y, and each processor contains a few kx-kz planes.

\newpage
\paragraph{Equations used:}
\begin{align}
	\paren{\frac{\p^2}{\p x^2}+\frac{\p^2}{\p z^2}} \omega_3  &= -\bracket{\frac{\p^2 }{\p y \p z} \omega_2 - \frac{\p}{\p x}\paren{\nabla^2 v}} \label{eq:om3}\\
	\paren{\frac{\p^2}{\p x^2}+\frac{\p^2}{\p z^2}} \omega_1  &= -\bracket{\frac{\p^2 }{\p x \p y } \omega_2 + \frac{\p}{\p z}\paren{\nabla^2 v}} \label{eq:om1}\\
	\paren{\frac{\p^2}{\p x^2}+\frac{\p^2}{\p z^2}} w  &= -\bracket{\frac{\p^2 }{\p y \p z} v + \frac{\p}{\p x}\omega_2} \label{eq:w}\\
	\paren{\frac{\p^2}{\p x^2}+\frac{\p^2}{\p z^2}} u  &= -\bracket{\frac{\p^2 }{\p x \p y} v - \frac{\p}{\p z}\omega_2} \label{eq:u}
\end{align}

\begin{table}[H]
	\centering 
	\renewcommand{\arraystretch}{1.5} 
	\begin{tabular}{l|l}
		Description& Line \\ \hline
		\textbf{for} each $y$ plane \textbf{do} &1100\\
		\qquad compute \textbf{o1c} ($\omega_1$), \textbf{o3c} ($\omega_3$), except 00 mode, \eqref{eq:om3}\eqref{eq:om1}& 1104-1123 \\
		\qquad compute \textbf{u1c} ($u$), \textbf{u3c} ($w$), except 00 mode, \eqref{eq:w}\eqref{eq:u}& 1126-1145 \\
		\qquad unpack 00 modes of \textbf{u1c}, \textbf{u3c}, \textbf{o1c}, \textbf{o3c} from \textbf{work2}&1149-1153 \\
		\qquad copy data for \textbf{u2c}, \textbf{o2c} &1156-1161 \\ \hdashline
		\multicolumn{2}{c}{\qquad \textbf{Now, have data for $u,v,w,\omega_1,\omega_2,\omega_3$ for the current time step in spectral space}}\\ \hdashline
		\qquad statistics,  vorticity at wall: \textbf{Wx0}, \textbf{Wz0}, \textbf{WxL}, \textbf{WzL}, velocity gradient tensor&1191-1403 \\
		\qquad Save \textbf{u1c} ($u$), \textbf{u2c} ($v$), \textbf{u3c} ($w$), \textbf{o1c} ($\omega_1$), \textbf{o2c} ($\omega_2$), \textbf{o3c} ($\omega_3$) to buffer& 1405-1414\\
		\qquad IFT to physical space \textbf{u1c} $\rightarrow$\textbf{u1r} ...&1419- 1426\\
		\qquad more statistics&1436-1469\\
		\qquad CFL&1473-1492\\
		\qquad compute $\bm{u}\times\bm{\omega}$: \textbf{u1r} ($H_3$), \textbf{u2r} ($H_1$), \textbf{u3r} ($H_2$)&1507-1517\\
		\qquad FFT to Fourier space \textbf{u1r} $\rightarrow$\textbf{u1c} ...&1526-1528\\
		\qquad save 00 mode of forcing: \textbf{rf0u} = $H_1(0,0)$,	\textbf{rf0w} = $H_3(0,0)$&1533-1534\\
		\qquad compute \textbf{ome1c} = $\frac{\p H_1}{\p x} + \frac{\p H_3}{\p z}$, \textbf{rhgc} =  $-\frac{\p H_3}{\p x} + \frac{\p H_1}{\p z}$, \textbf{rhvc} = $\paren{\frac{\p^2}{\p x^2}+\frac{\p^2}{\p z^2}} H_2$&1540-1563\\
		\qquad Save vorticity forcing \textbf{rhgc} ($h_g$) to buffer&1571-1574\\
		\textbf{END For}& 1577\\
		
		more statistics&1586-1595\\
		Update \textbf{Deltat} and \textbf{dtr}&1599-1617\\
		exchange \textbf{rf0u} and \textbf{rf0w} with all processors, with NaN detection before and after&1626-1681\\\hline
 	\end{tabular} 
\end{table}
\newpage
\subsection{cross1}
Defined on line 1 - 961 in cross.f

Used in main

Used for performing the time stepping

\begin{table}[H]
	\centering 
	\renewcommand{\arraystretch}{1.5} 
	\begin{tabular}{l|l}
		Description& Line \\ \hline
		\textbf{for} each time step \textbf{do} &179\\
		\qquad velocity gradient tensor and visualization (old code)&195-218\\
		\qquad \textbf{If} the previous step saved data to buffer, then write data to h5 file& 221-234\\
		\qquad \textbf{If} condition for \textbf{istep} or \textbf{time} is satisfied, change \textbf{collectFlowfield} to 1&235-239\\
		\qquad \textbf{If} the previous step is multiplier of \textbf{nimag}, then write restart file& 242-383\\ \hdashline
		
		\qquad \textbf{If} first time step: &389\\
		\qquad \qquad  Set BC&400-407\\
		\qquad \qquad Compute \textbf{hg}: $v$ and \textbf{hv}: $dv/dy$ from \textbf{phi}: $\phi$ using \textbf{Lapvdv}&410-441\\
		\qquad \qquad Copy \textbf{u00wk} = \textbf{u00}, \textbf{w00wk} = \textbf{w00}&444-450\\
		\qquad \qquad Compute $y$ derivative \textbf{rf0u}: ${\tfrac{d}{dy} u(0,0) = - \omega_3(0,0)}$, \textbf{rf0w}: ${\tfrac{d}{dy} w(0,0) = + \omega_1(0,0)}$&451-455\\
		\qquad \qquad Copy \textbf{vorwk} = \textbf{vor}: $\omega_2$, \textbf{phiwk} = \textbf{phi}: $\phi$&458-472\\
		\qquad \textbf{END If} for special first step&492\\ \hdashline
		
		\qquad \textbf{for} 3 RK steps \textbf{do}& 501\\
		\qquad \qquad Compute $y$ derivative \textbf{dvordy}: $d \omega_2 /dy$&504-508\\
		\qquad \qquad Change \textbf{phiwk}, \textbf{hv}, \textbf{hg}, \textbf{dvordy}, \textbf{vorwk} from $k_x-y$ planes to $k_x-k_z$ planes&511-518\\
		\qquad \qquad Stack 00 modes of $-\omega_3,\omega_1,u,w$ in \textbf{work}&521-539\\
		\qquad \qquad Compute \textbf{hv}: $\paren{\tfrac{\p^2}{\p x^2}+\tfrac{\p^2}{\p z^2}}H_2$, \textbf{hg}: $-\tfrac{\p H_3}{\p x} + \tfrac{\p H_1}{\p z}$, \textbf{dvordy}: $\tfrac{\p H_1}{\p x} + \tfrac{\p H_3}{\p z}$&\multirow{2}{*}{542-559}\\
		\qquad \qquad Compute \textbf{rf0u}: $H_1(k_x=k_z=0)$, \textbf{rf0w}: $H_3(k_x=k_z=0)$ with \textbf{hvhg}&\\
		\qquad \qquad Mean subtract for \textbf{rf0u} and \textbf{rf0w} \red{Problem!}&562-582\\
		\qquad \qquad Change \textbf{dvordy}, \textbf{hv}, \textbf{hg} from $k_x-k_z$ planes to $k_x-y$ planes&585-593\\
		\qquad \qquad Compute $y$ derivative \textbf{dvordy}:  $\frac{\p}{\p y} \paren{\tfrac{\p H_1}{\p x} + \tfrac{\p H_3}{\p z}}$&596-601\\
		\qquad \qquad Compute \textbf{hv}:  $\paren{\tfrac{\p^2}{\p x^2}+\tfrac{\p^2}{\p z^2}}H_2-\frac{\p}{\p y} \paren{\tfrac{\p H_1}{\p x} + \tfrac{\p H_3}{\p z}}$&604-616\\
		\qquad \qquad Save velocity forcing \textbf{hv} ($h_v$) to buffer&619-629
		\end{tabular} 
\end{table}
\newpage
\begin{table}[H]
	\centering 
	\renewcommand{\arraystretch}{1.5} 
	\begin{tabular}{l|l}
		\qquad \qquad Compute RHS forcing for $u_{00}$: \textbf{rf0u}; for $w_{00}$: \textbf{rf0w}; for $\omega_3$: \textbf{hg}; for $\phi$: \textbf{hv}&\multirow{2}{*}{632-669}\\ 
		\qquad \qquad \eqref{eq:u00p},\eqref{eq:w00p},\eqref{eq:ome},\eqref{eq:phip}&\\
		\qquad \qquad Set BC&672-672\\
		\qquad \qquad Compute and save wall acceleration to buffer&679-712\\
		\qquad \qquad Advance \textbf{phiwk}: $\phi$, \textbf{hg}: $v$, \textbf{hv}: $dv/dy$, \textbf{vorwk}: $\omega_2$ using \textbf{lapsov}&715-762\\
		\qquad \qquad Advance \textbf{u00wk}: $u_{00}$, \textbf{w00wk}: $w_{00}$ with constant mass flux using \textbf{Lapv1}&765-810\\
		\qquad \qquad Compute $y$ derivative \textbf{rf0u}: ${\tfrac{d}{dy} u(0,0) = - \omega_3(0,0)}$, \textbf{rf0w}: ${\tfrac{d}{dy} w(0,0) = + \omega_1(0,0)}$&812-821\\
		\qquad \textbf{END for} end loop over 3 RK stages&824\\ \hdashline
		
		\qquad Copy variable for the new time step:& \multirow{2}{*}{830-846}\\
		\qquad \textbf{u00} = \textbf{u00wk}, \textbf{w00} = \textbf{w00wk}, \textbf{vor} = \textbf{vorwk}, \textbf{phi} = \textbf{phiwk}&\\
		\qquad Master processor write history record& 849-915\\
		\qquad Increment Time&918-921\\
		\textbf{END for} time loop complete&943\\ \hline
	\end{tabular} 
\end{table}


Note: matrices \textbf{phi}, \textbf{vor}, \textbf{phiwk}, \textbf{vorwk}, \textbf{dvordy}, \textbf{hv}, \textbf{hg}, which are matrices that involves a change between $k_x-k_z$ plane and $k_x-y$ planes are allocated in main.f with sizes \textbf{nbuffsize}, and passed to cross1 as input. \textbf{nbuffsize} = mx*max(mmy*mz, mmz*my), which is sufficiently sized for data in both configurations.


\section{DNS Time}
\begin{figure}[H]
	\centering
	\includegraphics[width=0.75\textwidth]{DNSTime.png}
\end{figure}

At $istep = n$, the code propagates from $t = (n-1)\Delta t$ to $t = n \Delta t$, all the data written at this step, including restart, h5, and txt file are all data at $t = (n-1)\Delta t$. 

\begin{figure}[H]
	\centering
	\includegraphics[width=0.6\textwidth]{DNSSaving.png}
\end{figure}
\newpage
For run parameters:\\
$nstep = 10001$,\\
$nimag = 10000$, (save restart at $istep = 10001$)\\
save h5 at $istep = 50k \text{ or } 50k+1, k = 1,2,\dots,200$

First Run:\\
Start with $t = 0$\\
Snapshots saved at: $[49, 99, \dots, 9999] \Delta t$ and $[50, 100, \dots, 10000] \Delta t$\\
Restart saved at: $10000\Delta t$\\
Last line of txt file: $istep = 10001, t = 10000\Delta t$

Second Run:\\
Start with $t = 10000\Delta t$\\
Snapshots saved at: $[10049, 10099, \dots, 19999] \Delta t$ and  $[10050, 10100, \dots, 20000] \Delta t$\\
Restart saved at: $20000\Delta t$\\
Last line of txt file: $istep = 10001, t = 10000\Delta t$





\end{document}


